%%%%%%%%%%%%%%%%%%%%%%%%%%%%%%%%%%%%%%%%%%%%%%%%%%%%%%%%%%%%%%%%%%%%%%%%%%%%%%%%
%%% Results
%%%%%%%%%%%%%%%%%%%%%%%%%%%%%%%%%%%%%%%%%%%%%%%%%%%%%%%%%%%%%%%%%%%%%%%%%%%%%%%%

\clearpage
\section{Discussion}
\label{sec:discussion}

In this section I will describe the results of my research and development.
I will first describe the final product of my development.
Then I will list the numerical data from the machine learning experiments.
\nyi{User satisfaction}
Lastly I will examine the answers for the research questions and
evaluate the value of the thesis.

\subsection{Results}

%T\"ass\"a osassa esitet\"a\"an tulokset ja vastataan tutkielman alussa
%esitettyihin tutkimuskysymyksiin. Tieteellisen kirjoitelman
%arvo mitataan t\"ass\"a osassa esitettyjen tulosten perusteella.

\begin{figure}[htb]
\centering \includegraphics[width=\linewidth]{gfx/timeline.png}
\caption{Timeline view \label{fig:timeline}}
\end{figure}

\begin{figure}[htb]
\centering \includegraphics[width=\linewidth]{gfx/graph.png}
\caption{Graph view \label{fig:graph}}
\end{figure}

\begin{figure}[htb]
\centering \includegraphics[width=\linewidth]{gfx/estimates.jpg}
\caption{Timeline view with estimates \label{fig:estimates}}
\end{figure}

Figure \ref{fig:timeline} shows the final product developed in this thesis.
Figure \ref{fig:graph} shows the alternate directed graph view.
Figure \ref{fig:estimates} displays the timeline view with the future estimates. The red line describes the current time.

\begin{table}[htb]
\begin{center}
\begin{tabularx}{\linewidth}{| X | r | r |}
\hline
Dataset & Mean Absolute Error & RMSE \\
\hline
\textbf{TP50 value (median)} &  & \\
JSON template                       & 573.3 & 5871.4 \\
Automatically generated template    & 919.6 & 9937.3 \\
\hline
\textbf{TP75 value} &  & \\
JSON template                       & 710.5 & 5484.2 \\
Automatically generated template    & 954.0 & 9904.8 \\
\hline
\end{tabularx}
\end{center}
\caption{Results from plain statistics}
\label{tab:statresults}
\end{table}

To evaluate the results of the machine learning models, I first calculated a baseline for comparison.
The baseline consists of the same simple statistics as were used for the estimates in the application, the TP50 (median) and the TP75.
I calculated these two values for each transition (graph edge) based on the training set to construct a simple model. 
In the model each transition corresponds to a single value (the TP50 or the TP75). 
When predicting, the model always returns this value based on the transition in question.
Table \ref{tab:statresults} lists the results from this baseline model. 
The values of \textit{mean absolute error} and \textit{root mean square error} are listed.

I used the same training and testing sets to find the same error values for each machine learning model.
The error values are listed in table \ref{tab:mlresults}.

\begin{table}[htb]
\begin{center}
\begin{tabularx}{\linewidth}{| X | r | r | r | r |}
\hline
~ & \multicolumn{2}{c|}{BDT} & \multicolumn{2}{c|}{Poisson} \\
Dataset & Absolute & RMSE & Absolute & RMSE \\
\hline
\textbf{JSON template} &  &  &  &  \\
No extra features                   & 809.1 & 6290.7 & 881.1 & 5970.6 \\
with current time                   & 825.2 & 6130.3 & 881.1 & 5971.0 \\
with manual review                  & 689.7 & 5819.0 & 809.3 & 5800.9 \\
with resubmission                   & 831.5 & 6255.3 & 881.1 & 5971.2 \\
with manual review and resubmission & 703.7 & 5807.5 & 808.9 & 5800.8 \\
with all extra features             & 693.5 & 5801.9 & 808.3 & 5800.4 \\
\hline
\textbf{Automatically generated template} &  &  &  &  \\
No extra features                   & 994.8 & 5756.3 & 1243.9 & 6563.3 \\
with current time                   & 1099.7 & 6470.5 & 1225.8 & 6558.7 \\
with manual review                  & 815.1 & 5332.1 & 1038.5 & 5785.0 \\
with resubmission                   & 1004.3 & 5845.3 & 1219.1 & 6553.8 \\
with manual review and resubmission & 806.5 & 5428.4 & 1038.4 & 5785.7 \\
with all extra features             & 787.8 & 5234.4 & 1038.6 & 5785.4 \\
\hline
\end{tabularx}
\end{center}
\caption{Results from the best ML models}
\label{tab:mlresults}
\end{table}

\nyi{Talk about user satisfaction here?}

\subsection{Evaluation}
\label{sec:evaluation}

%Tutkimustuloksien merkityst\"a on aina syyt\"a arvioida ja tarkastella
%kriittisesti.  Joskus tarkastelu voi olla t\"ass\"a osassa, mutta se
%voidaan my\"os j\"att\"a\"a viimeiseen osaan, jolloin viimeisen osan nimeksi
%tulee >>Tarkastelu>>. Tutkimustulosten merkityst\"a voi arvioida my\"os
%>>Johtop\"a\"at\"okset>>-otsikon alla viimeisess\"a osassa. 

%T\"ass\"a osassa on syyt\"a my\"os arvioida tutkimustulosten luotettavuutta.
%Jos tutkimustulosten merkityst\"a arvioidaan >>Tarkastelu>>-osassa,
%voi luotettavuuden arviointi olla my\"os siell\"a. 

%\item How can a real-time event log from multiple sources and multiple concurrent workflows be dynamically transformed into a directed graph that describes the process? 
%\item How can the graph and past log data be used to predict future events and their times?
%\item How can the predictions be used to detect anomalies in new events (or lack thereof)?

\nyi{Talk about success with graph generation}\\
\nyi{notifications and their use}\\
\nyi{Changes in the system reflected real time (also a strength?)}\\
\nyi{Finding bugs in system (race conditions!) based on graphs}\\
\nyi{Feedback from users}

\nyi{Talk about negatives with machine learning models}\\
\nyi{Reason why they didn't end up working}

\nyi{Talk about how the system went immediately into production}\\
\nyi{Talk about the success of notifications built on top of the system}

% conforming to the JSON graphs (visually?)
% alternative graph?

%%%%%%%%%%%%%%%%%%%%%%%%%%%%%%%%%%%%%%%%%%%%%%%%%%%%%%%%%%%%%%%%%%%%%%%%%%%%%%%%

\subsection{Conclusions and Future Work} 
\label{sec:conclusions}

\begin{itemize}
\item[--]\nyi{Contributions (positives)}
\item[--]\nyi{Limitations (negatives)}
\item[--]\nyi{Future work}
\end{itemize}