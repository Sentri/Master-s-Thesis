%% using pdflatex, which directly typesets your document in
%% pdf (use jpg or pdf figures)
\documentclass[english,12pt,a4paper,pdftex,sci,utf8]{aaltothesis}
%% To the \documentclass above
%% specify your school: arts, biz, chem, elec, eng, sci
%% specify the character encoding scheme used by your editor: utf8, latin1

\usepackage{graphicx}

%% Use this if you write hard core mathematics, these are usually needed
\usepackage{amsfonts,amssymb,amsbsy}

\usepackage{csquotes}
\usepackage{biblatex}
\addbibresource{bibliography.bib}

%% Use the macros in this package to change how the hyperref package below 
%% typesets its hypertext -- hyperlink colour, font, etc. See the package
%% documentation. It also defines the \url macro, so use the package when 
%% not using the hyperref package.
%%
%\usepackage{url}

%% Use this if you want to get links and nice output. Works well with pdflatex.
\usepackage{hyperref}
\hypersetup{pdfpagemode=UseNone, pdfstartview=FitH,
  colorlinks=true,urlcolor=red,linkcolor=blue,citecolor=black,
  pdftitle={Default Title, Modify},pdfauthor={Your Name},
  pdfkeywords={Modify keywords}}

\PackageWarning{aaltothesis}{Remember to edit hypersetup tag}

%% All that is printed on paper starts here
\begin{document}

%% ONLY FOR M.Sc. AND LICENTIATE THESIS: Specify your department,
%% professorship and professorship code. 
%%
\department{Department of Computer Science}
\professorship{x}
%%


%% Choose one of these:
\univdegree{MSc}

%% Your own name (should be self explanatory...)
\author{Teemu Vartiainen}

%% Your thesis title comes here and again before a possible abstract in
%% Finnish or Swedish . If the title is very long and latex does an
%% unsatisfactory job of breaking the lines, you will have to force a
%% linebreak with the \\ control character. 
%% Do not hyphenate titles.

% working title
\thesistitle{Analyzing real-time and past events for describing processes and detecting anomalies}

\place{Espoo}

%% For B.Sc. thesis use the date when you present your thesis. 
%% 
%% Kandidaatintyön päivämäärä on sen esityspäivämäärä! 
\date{x.x.2017}

%% B.Sc. or M.Sc. thesis supervisor 
%% Note the "\" after the comma. This forces the following space to be 
%% a normal interword space, not the space that starts a new sentence. 
%% This is done because the fullstop isn't the end of the sentence that
%% should be followed by a slightly longer space but is to be followed
%% by a regular space.
%%
\supervisor{Prof.\ Petri Vuorimaa}

%% B.Sc. or M.Sc. thesis advisors(s). You can give upto two advisors in
%% this template. Check with your supervisor how many official advisors
%% you can have.
%%
%\advisor{Prof.\ Pirjo Professori}
\advisor{M.Sc.\ Tao Zhu}
\advisor{M.Sc.\ Rafael Forsbach Valle}

%% Aalto logo: syntax:
%% \uselogo{aaltoRed|aaltoBlue|aaltoYellow|aaltoGray|aaltoGrayScale}{?|!|''}
%%
%% Logo language is set to be the same as the document language.
%% Logon kieli on sama kuin dokumentin kieli
%%
\uselogo{aaltoBlue}{?}

%% Create the coverpage
%%
\makecoverpage

%%%%%%%%%%%%%%%%%%%%%%%%%%%%%%%%%%%%%%%%%%%%%%%%%%%%%%%%%%%%%%%%%%%%%%%%%%%%%%%%
%%% Abstracts & etc
%%%%%%%%%%%%%%%%%%%%%%%%%%%%%%%%%%%%%%%%%%%%%%%%%%%%%%%%%%%%%%%%%%%%%%%%%%%%%%%%

%% Note that when writting your master's thesis in English, place
%% the English abstract first followed by the possible Finnish abstract

%% English abstract.
%% All the information required in the abstract (your name, thesis title, etc.)
%% is used as specified above.
%% Specify keywords
%%
%% Kaikki tiivistelmässä tarvittava tieto (nimesi, työnnimi, jne.) käytetään
%% niin kuin se on yllä määritelty.
%% Avainsanat
%%
\keywords{event logs, machine learning, process mining}
%% Abstract text
\begin{abstractpage}[english]
  Your abstract in English. Try to keep the abstract short; approximately 
  100 words should be enough. The abstract explains your research topic, 
  the methods you have used, and the results you obtained.  
  Your abstract in English. Try to keep the abstract short; approximately 
  100 words should be enough. The abstract explains your research topic, 
  the methods you have used, and the results you obtained.  

  Your abstract in English. Try to keep the abstract short; approximately 
  100 words should be enough. The abstract explains your research topic, 
  the methods you have used, and the results you obtained.  
  Your abstract in English. Try to keep the abstract short; approximately 
  100 words should be enough. The abstract explains your research topic, 
  the methods you have used, and the results you obtained.  
\end{abstractpage}

%% Force a new page so that the possible English abstract starts on a new page
\newpage

%% Abstract in Finnish.

\thesistitle{Reaaliaikaisten ja menneiden prosessitapahtumien k\"aytt\"o prosessien kuvaamisessa}
\supervisor{Prof.\ Petri Vuorimaa}
\advisor{M.Sc.\ Tao Zhu}
\advisor{M.Sc.\ Rafael Forsbach Valle}
\department{Tietotekniikan laitos}
\professorship{x}
%% Avainsanat
\keywords{Tapahtumalokit, koneoppiminen, prosessikaaviot}
%% Tiivistelmän tekstiosa
\begin{abstractpage}[finnish]
  Tiivistelmässä on lyhyt selvitys (noin 100 sanaa)
  kirjoituksen tärkeimmästä sisällöstä: mitä ja miten on tutkittu,
  sekä mitä tuloksia on saatu. 
  Tiivistelmässä on lyhyt selvitys (noin 100 sanaa)
  kirjoituksen tärkeimmästä sisällöstä: mitä ja miten on tutkittu,
  sekä mitä tuloksia on saatu. 

  Tiivistelmässä on lyhyt selvitys (noin 100 sanaa)
  kirjoituksen tärkeimmästä sisällöstä: mitä ja miten on tutkittu,
  sekä mitä tuloksia on saatu. 
  Tiivistelmässä on lyhyt selvitys (noin 100 sanaa)
  kirjoituksen tärkeimmästä sisällöstä: mitä ja miten on tutkittu,
  sekä mitä tuloksia on saatu. 
  Tiivistelmässä on lyhyt selvitys (noin 100 sanaa)
  kirjoituksen tärkeimmästä sisällöstä: mitä ja miten on tutkittu,
  sekä mitä tuloksia on saatu. 
\end{abstractpage}

%% Preface
\mysection{Preface}

This thesis work was carried out at the Microsoft headquarters at the campus in Redmond, USA.
First I would like to thank...

% thank professor
% thank advisors for helping
% thank the team mates
% thank the company for the opportunity
% thank kara for unending support
% thank other friends

\vspace{5cm}
Espoo, y.y.2017

\vspace{5mm}
{\hfill Teemu T.\ Vartiainen \hspace{1cm}}

%% Force new page after preface
\newpage


%% Table of contents. 
\thesistableofcontents


%% Symbols and abbreviations
\mysection{Symbols and abbreviations placeholder}

\subsection*{Symbols}

\begin{tabular}{ll}
$\emptyset$  & empty set \\
\end{tabular}

\subsection*{Operators}

\begin{tabular}{ll}
$A \cup B$    & union of sets $A$ and $B$ \\
\end{tabular}

\subsection*{Abbreviations}

\begin{tabular}{ll}
SLA         & service level agreement \\
\end{tabular}


%% Tweaks the page numbering to meet the requirement of the thesis format:
%% Begin the pagenumbering in Arabian numerals (and leave the first page
%% of the text body empty, see \thispagestyle{empty} below).
%% Additionally, force the actual text to begin on a new page with the 
%% \clearpage command.
%% \clearpage is similar to \newpage, but it also flushes the floats (figures
%% and tables).
%% There is no need to change these
%%
\cleardoublepage
\storeinipagenumber
\pagenumbering{arabic}
\setcounter{page}{1}

%%%%%%%%%%%%%%%%%%%%%%%%%%%%%%%%%%%%%%%%%%%%%%%%%%%%%%%%%%%%%%%%%%%%%%%%%%%%%%%%
%%%%%%%%%%%%%%%%%%%%%%%%%%%%%%%%%%%%%%%%%%%%%%%%%%%%%%%%%%%%%%%%%%%%%%%%%%%%%%%%
%%% Content starts here
%%%%%%%%%%%%%%%%%%%%%%%%%%%%%%%%%%%%%%%%%%%%%%%%%%%%%%%%%%%%%%%%%%%%%%%%%%%%%%%%
%%%%%%%%%%%%%%%%%%%%%%%%%%%%%%%%%%%%%%%%%%%%%%%%%%%%%%%%%%%%%%%%%%%%%%%%%%%%%%%%

\section{Introduction}
\thispagestyle{empty}

\subsection{Motivation}
%       Importance of accurate real time data
\emph{Still somewhat placeholder-y} \\

Having diagnostics and real-time data is important in the modern software operations.
Dashboards, visualizations, and interactive logging is becoming more and more important
in the fast-changing digital world. Often these solutions are labeled Business Intelligence.
Having an accurate sense of current system status and low response time to faults can be a clear advantage
in business. Complex, distributed, and interconnected systems may make this kind of monitoring complex.

%       Importance of having both a big picture and detailed view of current processes
%         Information tailored to different people
Furthermore, there are often many stakeholders interested in this information. Thus, having customizable views
for this data can be crucial. Different people are interested in different granularities of information,
in addition to which parts of the data they want to see. The users need increasingly more tailored 
views to their data.

%       Predicting future events to give information and schedule tasks
In addition to views to the present, a peek to the future can provide a major advantage.
Being able to predict what happens soon in the future helps with scheduling tasks and lets people react 
to events before they even happen. With technologies like machine learning we can learn patterns in data and 
start estimating.

%       Loosely coupling diagnostics from systems
%       Adapting to change in agile and quickly changing environment
Another aspect of software operations is that they are ever-changing. 
Todays bleeding edge is tomorrow's obsolete. A tightly coupled diagnostics system
will soon become useless when the system or a workflow changes. Any analytics or visualization solution should
be agnostic to what the system actually does. This keeps the solution relevant for a long lifetime, and enables it
to be useful for different systems in different environments. 
Furthermore, the solution should adapt to change seamlessly without constant maintenance from the users or the system administrators. However, unintended changes should be detected and notified about.


\subsection{Microsoft}
%        store with tons of products and parallel workflows ongoing continuously
At a multinational company as large as Microsoft the scale of information is staggering.
Distributed systems process data and generate log entries concurrently in the numbers of terabytes.
No single person can monitor all the logs from even a single system without the help of automated
graphs, alerts, and notifications.  

%        many different users need different kinds of information
%        third party developers and customers want to know what is going on
These system log entries are collected as they happen and stored. Different stakeholders from engineers to third-party developers to retail users may benefit from this data. However, the needs for each of them are
different with regards to detail and types of information \emph{(elaborate?)}. Furthermore, issues of privacy and 
confidentiality further complicate this issue.

%            we need to be able to answer and provide a good service
When an issue arises, engineers and managers need to have tools to investigate what is happening. 
In some cases, when a system workflow is started, the status can be unknown until it completes.
The duration can vary from minutes to hours to days. This kind of workflow can be seen as a black box,
when it is hard to determine further information. This is where an intelligent logging and visualization tool is 
needed. A solution that provides both the big picture in addition to details will help investigate issues and
provide good service to third parties or customers.

%        detecting failures is essential to keep up SLAs and good experience
%            automation, prioritization, customization
In addition, when a company such as Microsoft is providing a service, they have to adhere to contracts such as
service level agreements (SLA). These contracts determine maximum response times and durations for various processes.
Prioritizing tasks and alerting about faults as early as possible allows for good service. Furthermore, these
notifications should be highly customizable so they go to the right people at the right time.

\subsection{Research goals}
%		Questions and elaboration
%       Scope and main concepts of research
In this thesis I investigate methods of monitoring concurrent and distributed systems which are running independent workflows with distinct steps. These workflows are triggered based on real world events. They start and stop at undetermined times and their running times depend on the task at hand and the system load. 
The thesis investigates analyzing these workflows in real time and how to provide useful information to different users with different needs. 

I carried our research and development at Microsoft to answer the following research questions:

\begin{enumerate}
    \item How can a real-time event log from multiple sources and multiple concurrent workflows be dynamically transformed into a directed graph that describes the process? 
    \item How can the graph and past log data be used to predict future events and their times?
    \item How can the predictions be used to detect anomalies in new events (or lack thereof)?
\end{enumerate}

\emph{(Elaborate scope further)}

\subsection{Structure of the Thesis}
%   d. Structure of thesis
This thesis is divided in ... chapters. Chapter x explains ... and after that ... 
\emph{(TBD)}

%% In a thesis, every section starts a new page, hence \clearpage
\clearpage
\section{Background}
\label{sec:background}

\begin{itemize}
\item[--]Process mining
\item[--]Machine learning
\item[--]Anomaly detection
\item[--]..?
\end{itemize}

\clearpage
\section{Related work}
\label{sec:relatedwork}

Petri nets from event logs \cite{van2013discovering}.
Anomaly detection \cite{bezerra2009anomaly}.

\clearpage
\section{Problem statement}
\label{sec:problem}

% description about workflows
%  how they are similar but also different
%  multiple concurrent flows
%  different but unknown forks and parallelized steps
%  durations unknown

% real time events from different sources
% listeners interpreting events
% master log containing everything
% log entries can be queried from this master log
% order unknown, somewhat accurate timestamps

% user needs
%  visualize processes without supervision or configuration
%  unknown parallelism involved
%  must adapt to changes in workflow

% two sides of the information
%  aggregate for past data
%  real-time info about current processes

%  different users concerned about different workflow steps or different entities

\clearpage
\section{Research material and methods}
\label{sec:methods}

%T\"ass\"a osassa kuvataan k\"aytetty tutkimusaineisto ja
%tutkimuksen metodologiset valinnat, sek\"a
%kerrotaan tutkimuksen toteutustapa ja k\"aytetyt menetelm\"at. 

\begin{itemize}
\item[--]Data source/description
\item[--]Process mining methods
\item[--]ML models and methods
\end{itemize}

\clearpage
\section{Implementation details}

\begin{itemize}
\item[--]Setup, DB, caching
\item[--]Graph construction
\item[--]Timeline construction
\item[--]Prediction
\item[--]ML
\item[--]Anomalies/health detection
\end{itemize}

\clearpage
\section{Results}
\label{sec:results}

%T\"ass\"a osassa esitet\"a\"an tulokset ja vastataan tutkielman alussa
%esitettyihin tutkimuskysymyksiin. Tieteellisen kirjoitelman
%arvo mitataan t\"ass\"a osassa esitettyjen tulosten perusteella.

\begin{itemize}
\item[--]Generated process graphs
\item[--]Prediction accuracy (numbers)
\item[--]ML scores (numbers, graphs)
\end{itemize}

\clearpage
\section{Evaluation}
\label{sec:evaluation}

%Tutkimustuloksien merkityst\"a on aina syyt\"a arvioida ja tarkastella
%kriittisesti.  Joskus tarkastelu voi olla t\"ass\"a osassa, mutta se
%voidaan my\"os j\"att\"a\"a viimeiseen osaan, jolloin viimeisen osan nimeksi
%tulee >>Tarkastelu>>. Tutkimustulosten merkityst\"a voi arvioida my\"os
%>>Johtop\"a\"at\"okset>>-otsikon alla viimeisess\"a osassa. 

%T\"ass\"a osassa on syyt\"a my\"os arvioida tutkimustulosten luotettavuutta.
%Jos tutkimustulosten merkityst\"a arvioidaan >>Tarkastelu>>-osassa,
%voi luotettavuuden arviointi olla my\"os siell\"a. 

\clearpage
\section{Conclusions and future work} 
\label{sec:conclusions}

\begin{itemize}
\item[--]Contributions (positives)
\item[--]Limitations (negatives)
\item[--]Future work
\end{itemize}

%%%%%%%%%%%%%%%%%%%%%%%%%%%%%%%%%%%%%%%%%%%%%%%%%%%%%%%%%%%%%%%%%%%%%%%%%%%%%%%%
%%%%%%%%%%%%%%%%%%%%%%%%%%%%%%%%%%%%%%%%%%%%%%%%%%%%%%%%%%%%%%%%%%%%%%%%%%%%%%%%
%%% References
%%%%%%%%%%%%%%%%%%%%%%%%%%%%%%%%%%%%%%%%%%%%%%%%%%%%%%%%%%%%%%%%%%%%%%%%%%%%%%%%
%%%%%%%%%%%%%%%%%%%%%%%%%%%%%%%%%%%%%%%%%%%%%%%%%%%%%%%%%%%%%%%%%%%%%%%%%%%%%%%%

\clearpage
%% The \phantomsection command is necessary for hyperref to jump to the 
%% correct page, in other words it puts a hyper marker on the page.

\phantomsection
\addcontentsline{toc}{section}{References}

\printbibliography

%% Appendices
%\clearpage
%\thesisappendix

\end{document}
