%% using pdflatex, which directly typesets your document in
%% pdf (use jpg or pdf figures)
\documentclass[english,12pt,a4paper,pdftex,sci,utf8]{aaltothesis}

%% To the \documentclass above
%% specify your school: arts, biz, chem, elec, eng, sci
%% specify the character encoding scheme used by your editor: utf8, latin1


\usepackage{graphicx}

%% Use this if you write hard core mathematics, these are usually needed
\usepackage{amsfonts,amssymb,amsbsy}

\usepackage{csquotes}
\usepackage{biblatex}
\addbibresource{bibliography.bib}

%% Use the macros in this package to change how the hyperref package below 
%% typesets its hypertext -- hyperlink colour, font, etc. See the package
%% documentation. It also defines the \url macro, so use the package when 
%% not using the hyperref package.
%%
%\usepackage{url}

%% Use this if you want to get links and nice output. Works well with pdflatex.
\usepackage{hyperref}
\hypersetup{pdfpagemode=UseNone, pdfstartview=FitH,
  colorlinks=true,urlcolor=red,linkcolor=blue,citecolor=black,
  pdftitle={Default Title, Modify},pdfauthor={Your Name},
  pdfkeywords={Modify keywords}}


%% All that is printed on paper starts here
\begin{document}

%% ONLY FOR M.Sc. AND LICENTIATE THESIS: Specify your department,
%% professorship and professorship code. 
%%
\department{Department of Computer Science}
\professorship{x}
%%


%% Choose one of these:
\univdegree{MSc}

%% Your own name (should be self explanatory...)
\author{Teemu Vartiainen}

%% Your thesis title comes here and again before a possible abstract in
%% Finnish or Swedish . If the title is very long and latex does an
%% unsatisfactory job of breaking the lines, you will have to force a
%% linebreak with the \\ control character. 
%% Do not hyphenate titles.
%% 
\thesistitle{Working title: Real-time logs in process mining}

\place{Espoo}

%% For B.Sc. thesis use the date when you present your thesis. 
%% 
%% Kandidaatintyön päivämäärä on sen esityspäivämäärä! 
\date{x.x.2017}

%% B.Sc. or M.Sc. thesis supervisor 
%% Note the "\" after the comma. This forces the following space to be 
%% a normal interword space, not the space that starts a new sentence. 
%% This is done because the fullstop isn't the end of the sentence that
%% should be followed by a slightly longer space but is to be followed
%% by a regular space.
%%
\supervisor{Prof.\ Petri Vuorimaa}

%% B.Sc. or M.Sc. thesis advisors(s). You can give upto two advisors in
%% this template. Check with your supervisor how many official advisors
%% you can have.
%%
%\advisor{Prof.\ Pirjo Professori}
\advisor{M.Sc.\ Tao Zhu}
\advisor{M.Sc.\ Rafael Forsbach Valle}

%% Aalto logo: syntax:
%% \uselogo{aaltoRed|aaltoBlue|aaltoYellow|aaltoGray|aaltoGrayScale}{?|!|''}
%%
%% Logo language is set to be the same as the document language.
%% Logon kieli on sama kuin dokumentin kieli
%%
\uselogo{aaltoBlue}{?}

%% Create the coverpage
%%
\makecoverpage

%%%%%%%%%%%%%%%%%%%%%%%%%%%%%%%%%%%%%%%%%%%%%%%%%%%%%%%%%%%%%%%%%%%%%%%%%%%%%%%%
%%% Abstracts & etc
%%%%%%%%%%%%%%%%%%%%%%%%%%%%%%%%%%%%%%%%%%%%%%%%%%%%%%%%%%%%%%%%%%%%%%%%%%%%%%%%

%% Note that when writting your master's thesis in English, place
%% the English abstract first followed by the possible Finnish abstract

%% English abstract.
%% All the information required in the abstract (your name, thesis title, etc.)
%% is used as specified above.
%% Specify keywords
%%
%% Kaikki tiivistelmässä tarvittava tieto (nimesi, työnnimi, jne.) käytetään
%% niin kuin se on yllä määritelty.
%% Avainsanat
%%
\keywords{event logs, machine learning, process mining}
%% Abstract text
\begin{abstractpage}[english]
  Your abstract in English. Try to keep the abstract short; approximately 
  100 words should be enough. The abstract explains your research topic, 
  the methods you have used, and the results you obtained.  
  Your abstract in English. Try to keep the abstract short; approximately 
  100 words should be enough. The abstract explains your research topic, 
  the methods you have used, and the results you obtained.  

  Your abstract in English. Try to keep the abstract short; approximately 
  100 words should be enough. The abstract explains your research topic, 
  the methods you have used, and the results you obtained.  
  Your abstract in English. Try to keep the abstract short; approximately 
  100 words should be enough. The abstract explains your research topic, 
  the methods you have used, and the results you obtained.  
\end{abstractpage}

%% Force a new page so that the possible English abstract starts on a new page
\newpage

%% Abstract in Finnish.  Delete if you don't need it. 
\thesistitle{Tapahtumalokien käyttö prosessien kuvaamisessa}
\supervisor{Prof.\ Petri Vuorimaa}
\advisor{M.Sc. Tao Zhu}
\advisor{M.Sc. Rafael Forsbach Valle}
%\degreeprogram{Electronics and electrical engineering}
\department{Tietotekniikan laitos}
\professorship{x}
%% Avainsanat
\keywords{Tapahtumalokit, koneoppiminen, prosessikaaviot}
%% Tiivistelmän tekstiosa
\begin{abstractpage}[finnish]
  Tiivistelmässä on lyhyt selvitys (noin 100 sanaa)
  kirjoituksen tärkeimmästä sisällöstä: mitä ja miten on tutkittu,
  sekä mitä tuloksia on saatu. 
  Tiivistelmässä on lyhyt selvitys (noin 100 sanaa)
  kirjoituksen tärkeimmästä sisällöstä: mitä ja miten on tutkittu,
  sekä mitä tuloksia on saatu. 

  Tiivistelmässä on lyhyt selvitys (noin 100 sanaa)
  kirjoituksen tärkeimmästä sisällöstä: mitä ja miten on tutkittu,
  sekä mitä tuloksia on saatu. 
  Tiivistelmässä on lyhyt selvitys (noin 100 sanaa)
  kirjoituksen tärkeimmästä sisällöstä: mitä ja miten on tutkittu,
  sekä mitä tuloksia on saatu. 
  Tiivistelmässä on lyhyt selvitys (noin 100 sanaa)
  kirjoituksen tärkeimmästä sisällöstä: mitä ja miten on tutkittu,
  sekä mitä tuloksia on saatu. 
\end{abstractpage}

%% Preface
\mysection{Preface}
Placeholder: I want to thank somebody for something.\\

\vspace{5cm}
Espoo, y.y.2017

\vspace{5mm}
{\hfill Teemu T.\ Vartiainen \hspace{1cm}}

%% Force new page after preface
\newpage


%% Table of contents. 
\thesistableofcontents


%% Symbols and abbreviations
\mysection{Symbols and abbreviations placeholder}

\subsection*{Symbols}

\begin{tabular}{ll}
$\mathbf{B}$  & magnetic flux density  \\
\end{tabular}

\subsection*{Operators}

\begin{tabular}{ll}
$\mathbf{A} \cdot \mathbf{B}$    & dot product of vectors $\mathbf{A}$ and $\mathbf{B}$ \\
\end{tabular}

\subsection*{Abbreviations}

\begin{tabular}{ll}
AC         & alternating current \\
\end{tabular}


%% Tweaks the page numbering to meet the requirement of the thesis format:
%% Begin the pagenumbering in Arabian numerals (and leave the first page
%% of the text body empty, see \thispagestyle{empty} below).
%% Additionally, force the actual text to begin on a new page with the 
%% \clearpage command.
%% \clearpage is similar to \newpage, but it also flushes the floats (figures
%% and tables).
%% There is no need to change these
%%
\cleardoublepage
\storeinipagenumber
\pagenumbering{arabic}
\setcounter{page}{1}

%%%%%%%%%%%%%%%%%%%%%%%%%%%%%%%%%%%%%%%%%%%%%%%%%%%%%%%%%%%%%%%%%%%%%%%%%%%%%%%%
%%%%%%%%%%%%%%%%%%%%%%%%%%%%%%%%%%%%%%%%%%%%%%%%%%%%%%%%%%%%%%%%%%%%%%%%%%%%%%%%
%%% Content starts here
%%%%%%%%%%%%%%%%%%%%%%%%%%%%%%%%%%%%%%%%%%%%%%%%%%%%%%%%%%%%%%%%%%%%%%%%%%%%%%%%
%%%%%%%%%%%%%%%%%%%%%%%%%%%%%%%%%%%%%%%%%%%%%%%%%%%%%%%%%%%%%%%%%%%%%%%%%%%%%%%%

%% Text body begins. Note that since the text body
%% is mostly in Finnish the majority of comments are
%% also in Finnish after this point. There is no point in explaining
%% Finnish-language specific thesis conventions in English. Someday 
%% this text will possibly be translated to English.
%%
\section{Introduction}

%% Leave first page empty
\thispagestyle{empty}

This should be a 2-4 page introduction with the following things:

%% Esimerkki luettelosta. Lyhyt ajatusviiva on k\"ayt\"oss\"a
%% luettelossa, ja se on pituudeltaan
%% en dash. Merkit\"a\"an latex-koodissa --. 

\begin{itemize}
\item[--]Research background and a general introduction to the topic
\item[--]Research goals
\item[--]Main research question and partial problems
\item[--]Scope and main concepts of research
\item[--]Quick structure of the thesis?
\end{itemize}

\subsection{Research questions placeholder}

\begin{itemize}
\item[a]How can a real-time event log from multiple sources and multiple concurrent workflows be dynamically transformed into a directed graph that describes the process? (process mining) 
\item[b]How can the graph and past log data be used to predict future events and their times? (ML models?) 
\item[c]How can the predictions be used to detect anomalies in new events?
\end{itemize}

%	a. Motivation
%		Logs, parallelism, processes, workflows, machine learning, real time, dynamic adaptation
%	b. Microsoft?
%	c. Research goals
%		Questions and elaboration
%   d. Structure of thesis

%% In a thesis, every section starts a new page, hence \clearpage
\clearpage
\section{Background}

\begin{itemize}
\item[--]Process mining
\item[--]Machine learning
\item[--]Anomaly detection
\item[--]..?
\end{itemize}

%% \section{osan otsikko} 
%% \subsection{alaotsikko}
%% \subsubsection{ala-alaotsikko}
%%
%% Three levels of hierarchy in sectioning should be enough

\clearpage
\section{Related work}

Petri nets from event logs \cite{van2013discovering}.
Anomaly detection \cite{bezerra2009anomaly}.

\clearpage
\section{Problem statement}

\begin{itemize}
\item[--]Event sources
\item[--]Logs
\item[--]Consumption
\item[--]User needs
\end{itemize}

\clearpage
\section{Research material and methods}

%T\"ass\"a osassa kuvataan k\"aytetty tutkimusaineisto ja
%tutkimuksen metodologiset valinnat, sek\"a
%kerrotaan tutkimuksen toteutustapa ja k\"aytetyt menetelm\"at. 

\begin{itemize}
\item[--]Data source/description
\item[--]Process mining methods
\item[--]ML models and methods
\end{itemize}

\clearpage
\section{Implementation details}

\begin{itemize}
\item[--]Setup, DB, caching
\item[--]Graph construction
\item[--]Timeline construction
\item[--]Prediction
\item[--]ML
\item[--]Anomalies/health detection
\end{itemize}

\clearpage
\section{Results}

%T\"ass\"a osassa esitet\"a\"an tulokset ja vastataan tutkielman alussa
%esitettyihin tutkimuskysymyksiin. Tieteellisen kirjoitelman
%arvo mitataan t\"ass\"a osassa esitettyjen tulosten perusteella.

\begin{itemize}
\item[--]Generated process graphs
\item[--]Prediction accuracy (numbers)
\item[--]ML scores (numbers, graphs)
\end{itemize}

\clearpage
\section{Evaluation}

%Tutkimustuloksien merkityst\"a on aina syyt\"a arvioida ja tarkastella
%kriittisesti.  Joskus tarkastelu voi olla t\"ass\"a osassa, mutta se
%voidaan my\"os j\"att\"a\"a viimeiseen osaan, jolloin viimeisen osan nimeksi
%tulee >>Tarkastelu>>. Tutkimustulosten merkityst\"a voi arvioida my\"os
%>>Johtop\"a\"at\"okset>>-otsikon alla viimeisess\"a osassa. 

%T\"ass\"a osassa on syyt\"a my\"os arvioida tutkimustulosten luotettavuutta.
%Jos tutkimustulosten merkityst\"a arvioidaan >>Tarkastelu>>-osassa,
%voi luotettavuuden arviointi olla my\"os siell\"a. 

\clearpage
\section{Conclusions and future work} 

\begin{itemize}
\item[--]Contributions (positives)
\item[--]Limitations (negatives)
\item[--]Future work
\end{itemize}

%%%%%%%%%%%%%%%%%%%%%%%%%%%%%%%%%%%%%%%%%%%%%%%%%%%%%%%%%%%%%%%%%%%%%%%%%%%%%%%%
%%%%%%%%%%%%%%%%%%%%%%%%%%%%%%%%%%%%%%%%%%%%%%%%%%%%%%%%%%%%%%%%%%%%%%%%%%%%%%%%
%%% References
%%%%%%%%%%%%%%%%%%%%%%%%%%%%%%%%%%%%%%%%%%%%%%%%%%%%%%%%%%%%%%%%%%%%%%%%%%%%%%%%
%%%%%%%%%%%%%%%%%%%%%%%%%%%%%%%%%%%%%%%%%%%%%%%%%%%%%%%%%%%%%%%%%%%%%%%%%%%%%%%%

\clearpage
%% The \phantomsection command is nessesary for hyperref to jump to the 
%% correct page, in other words it puts a hyper marker on the page.

\phantomsection
\addcontentsline{toc}{section}{References}

\printbibliography

%% Appendices
%% Liitteet
%\clearpage
%\thesisappendix

\end{document}
