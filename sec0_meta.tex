
%%%%%%%%%%%%%%%%%%%%%%%%%%%%%%%%%%%%%%%%%%%%%%%%%%%%%%%%%%%%%%%%%%%%%%%%%%%%%%%%
%%% Abstracts & etc
%%%%%%%%%%%%%%%%%%%%%%%%%%%%%%%%%%%%%%%%%%%%%%%%%%%%%%%%%%%%%%%%%%%%%%%%%%%%%%%%

\keywords{Process discovery, Machine learning, Event logs}

%% Abstract text
\begin{abstractpage}[english]
  Your abstract in English. Try to keep the abstract short; approximately 
  100 words should be enough. The abstract explains your research topic, 
  the methods you have used, and the results you obtained.  
  Your abstract in English. Try to keep the abstract short; approximately 
  100 words should be enough. The abstract explains your research topic, 
  the methods you have used, and the results you obtained.  

  Your abstract in English. Try to keep the abstract short; approximately 
  100 words should be enough. The abstract explains your research topic, 
  the methods you have used, and the results you obtained.  
  Your abstract in English. Try to keep the abstract short; approximately 
  100 words should be enough. The abstract explains your research topic, 
  the methods you have used, and the results you obtained.  
\end{abstractpage}

\newpage

%% Abstract in Finnish.

\thesistitle{Reaaliaikaisten ja menneiden prosessitapahtumien k\"aytt\"o prosessitutkinnassa ja poikkeavuuksien havainnassa}
\supervisor{Prof.\ Petri Vuorimaa}
\advisor{M.Sc.\ Tao Zhu}
\advisor{M.Sc.\ Rafael Forsbach Valle}
\department{Tietotekniikan laitos}
\professorship{x}
%% Avainsanat
\keywords{Prosessitutkinta, Koneoppiminen, Tapahtumalokit}
%% Tiivistelmän tekstiosa
\begin{abstractpage}[finnish]
  Tiivistelmässä on lyhyt selvitys (noin 100 sanaa)
  kirjoituksen tärkeimmästä sisällöstä: mitä ja miten on tutkittu,
  sekä mitä tuloksia on saatu. 
  Tiivistelmässä on lyhyt selvitys (noin 100 sanaa)
  kirjoituksen tärkeimmästä sisällöstä: mitä ja miten on tutkittu,
  sekä mitä tuloksia on saatu. 

  Tiivistelmässä on lyhyt selvitys (noin 100 sanaa)
  kirjoituksen tärkeimmästä sisällöstä: mitä ja miten on tutkittu,
  sekä mitä tuloksia on saatu. 
  Tiivistelmässä on lyhyt selvitys (noin 100 sanaa)
  kirjoituksen tärkeimmästä sisällöstä: mitä ja miten on tutkittu,
  sekä mitä tuloksia on saatu. 
  Tiivistelmässä on lyhyt selvitys (noin 100 sanaa)
  kirjoituksen tärkeimmästä sisällöstä: mitä ja miten on tutkittu,
  sekä mitä tuloksia on saatu. 
\end{abstractpage}

%% Preface
\mysection{Preface}

This thesis work was carried out at the Microsoft headquarters at the campus in Redmond, USA.
It has been written as the last part of my Master's degree.

I would like to thank the people at Microsoft for the opportunity.
Thank you Kristin and Britt at recruiting for figuring out the unusual circumstances for me.
Thank you Tao and Rafael for mentoring and guiding me during my work.
I would also like to thank the rest of the ICH Catalog Health team for dealing with my constant questions
and helping me feel like home in my team.

Great thanks to professor Petri Vuorimaa for being my supervisor and for helping me with the practicalities.

I would like to thank my friends for supporting me.
I would like to thank my girlfriend Kara for her unending support and understanding during this stressful time.
It is not an understatement that without you I would not be where I am today.
Thank you Max and David for your endless support and for proofreading.
Thank you Xane, Jorge, Cody, and other friends and birds for keeping me sane.

Lastly I would like to thank all my family for standing by me and dealing with my travels and schedules.

\vspace{5cm}
Espoo, y.y.2017

\vspace{5mm}
{\hfill Teemu T.\ Vartiainen \hspace{1cm}}

\newpage
\thesistableofcontents

\mysection{Symbols and Abbreviations placeholder}

\subsection*{Symbols}

\begin{tabular}{ll}
$\mathbb{B}(X)$  & a set of multi-sets over $X$ \\
$\emptyset$      & empty set \\
\end{tabular}

\subsection*{Operators}

\begin{tabular}{ll}
$a \in A$    & $a$ is a member of set $A$ \\
$A \cap B$   & intersection of sets $A$ and $B$ \\
\end{tabular}

\subsection*{Abbreviations}

\begin{tabular}{ll}
BDT         & boosted decision tree
BI          & business intelligence \\
MBI         & medium business intelligence \\
ML          & machine learning \\
MSE         & mean square error \\
PII         & personally identifiable information \\
SLA         & service level agreement \\
SQL         & structured query language \\
TP          & top percentile \\
\end{tabular}
